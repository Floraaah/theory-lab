\documentclass[11pt]{article}
\usepackage[a4paper,margin=1in]{geometry}
\usepackage{amsmath, amssymb, amsthm, mathtools}
\usepackage{setspace, microtype, titlesec, hyperref, xcolor, graphicx, enumitem}
\usepackage{lmodern}
\usepackage{quoting}
\quotingsetup{vskip=1em, leftmargin=2em, rightmargin=2em, font=itshape}
\usepackage{xeCJK}
\setCJKmainfont{Source Han Serif SC}
\usepackage{newtxtext,newtxmath} % Times-like

\usepackage{titlesec}
\titlespacing*{\section}{0pt}{1.5em}{0.7em}
\titlespacing*{\subsection}{0pt}{1.0em}{0.5em}

\setstretch{1.2}
\setlength{\parskip}{0.6em}

% 美观的 section 结构
\titleformat{\section}{\large\bfseries}{\thesection}{1em}{}
\titleformat{\subsection}{\normalsize\bfseries}{\thesubsection}{0.8em}{}
\hypersetup{
  colorlinks=true,
  linkcolor=blue,
  citecolor=teal,
  urlcolor=blue
}

% 数学符号
\newcommand{\F}{\mathcal{F}}
\newcommand{\projT}{\Pi_{T_A(\F)}}
\newcommand{\projN}{\Pi_{N_A(\F)}}

\begin{document}
\begin{center}
  {\LARGE\bfseries After Aghion and Acemoglu: From Narrative Growth to
Structural Feasibility \par}
  \vspace{0.5em}{\large Yufei Li}\par
  \vspace{0.3em}{\small Independent Researcher}\par
  \vspace{0.3em}{\small October 2025}\par
\end{center}
\vspace{1em}

Abstract

This note presents a formal transition from narrative dynamics to
structural feasibility in endogenous growth theory. It reinterprets
Aghion and Acemoglu's framework through a constraint-geometric lens,
replacing the optimization metaphor with a geometric one. While the
former constructs growth as a trade-off between innovation and policy
within a parameterized equilibrium, the latter redefines dynamics as
constrained motion on a manifold of feasible structures. The key step is
to replace the optimization metaphor with a geometric one: agents do not
maximize along a fixed frontier but evolve within the tangent space of a
constraint geometry whose curvature embodies the non-commutativity of
adaptation. This shift transforms ``growth'' from an aggregate process
to a generative condition of feasibility preservation. By articulating
this curvature--tension correspondence, the paper argues that the formal
economy can be understood as a dissipative yet self-stabilizing
system---an ontological continuation beyond the narrative closure of
macroeconomics.

\begin{otherlanguage}{chinese}

\subsection{中文导语}\label{ux4e2dux6587ux5bfcux8bed}

本笔记讨论``内生增长理论''在 Aghion--Acemoglu
框架中达到叙事完结(narrative
closure)后的结构封顶问题,并指出其形式延伸方向:由\textbf{参数化可行性(parametric
feasibility)}转向\textbf{约束几何可行性(constraint geometry of
feasibility)}。\\
核心思想是将``创新''从增长驱动变量转化为``约束变形过程'',并以结构自洽性而非最优性为系统稳定条件。\\
换言之,从``解释增长的故事''走向``生成增长的几何''。

\textbf{术语映射}

{\def\LTcaptype{} % do not increment counter
\begin{longtable}[]{@{}
  >{\raggedright\arraybackslash}p{(\linewidth - 4\tabcolsep) * \real{0.1364}}
  >{\raggedright\arraybackslash}p{(\linewidth - 4\tabcolsep) * \real{0.5758}}
  >{\raggedright\arraybackslash}p{(\linewidth - 4\tabcolsep) * \real{0.2879}}@{}}
\toprule\noalign{}
\begin{minipage}[b]{\linewidth}\raggedright
中文概念
\end{minipage} & \begin{minipage}[b]{\linewidth}\raggedright
英文概念
\end{minipage} & \begin{minipage}[b]{\linewidth}\raggedright
说明
\end{minipage} \\
\midrule\noalign{}
\endhead
\bottomrule\noalign{}
\endlastfoot
内生增长的叙事完结 & narrative closure of endogenous growth &
指创新与制度被完全内生化的宏观模型框架 \\
参数化可行性 & parametric feasibility & 模型通过外生参数假定系统可行 \\
约束几何 & constraint geometry & 表示系统内部约束关系的几何结构 \\
动态可行性 & dynamic feasibility & 可随时间演化且内部自洽的可行域 \\
生成性稳定 & generative stability & 系统能维持自身约束条件的平衡状态 \\
张力 & structural tension & 结构内在的约束冲突或对称破缺 \\
自生系统 & self-sustaining system & 通过内部约束反馈维持持续性的系统 \\
悔与失效 & regret and obsolescence & 动态约束再配置的表征机制 \\
\end{longtable}
}

\begin{quote}
本文延续``第三断裂后''的思维语境:\\
从叙事的增长(growth-as-story)转向结构的生长(growth-as-structure)。\\
它既是对宏观经济学第二幕的致敬,也为形式经济学的第三幕------生成稳定性------奠定语言坐标。
\end{quote}

\end{otherlanguage}

\begin{quote}
Bridging macroeconomic narratives and formal structural theory, this
note interprets Aghion and Acemoglu's innovation--institution framework
as the narrative completion of the second layer of growth theory. It
argues that the next abstraction should reconstruct the same logic
within a self-sustaining system of constraints---where feasibility,
rather than optimization, defines generative stability.
\end{quote}

\begin{center}\rule{0.5\linewidth}{0.5pt}\end{center}

\subsection{1. The Narrative Completion of Endogenous
Growth}\label{the-narrative-completion-of-endogenous-growth}

Aghion and Acemoglu jointly established the \textbf{narrative closure}
of modern macroeconomic theory:\\
innovation became endogenous, and institutions became adaptive.

Their frameworks---\emph{Schumpeterian growth} and \emph{institutional
dynamics}---achieved conceptual closure by turning ``technological
progress'' and ``political constraint'' into formal state variables.\\
This closure, however, also \textbf{stabilized the narrative tension}:
growth was explained as a sequence of parameter adjustments, not as a
self-generating constraint system.

\begin{quote}
In short, the endogenous growth literature perfected the story of change
without formalizing the structure of change.
\end{quote}

\begin{center}\rule{0.5\linewidth}{0.5pt}\end{center}

这里是改写后的 \textbf{§2 ``The Formal Limit of Parameterized
Narratives''},已校正了你对 \emph{Aghion--Howitt (1992)} 与
\emph{Acemoglu} 两类模型的技术描述,同时保持原文的哲学节奏与语言密度:

\begin{center}\rule{0.5\linewidth}{0.5pt}\end{center}

\subsection{2. The Formal Limit of Parameterized
Narratives}\label{the-formal-limit-of-parameterized-narratives}

The Schumpeterian model of Aghion and Howitt (1992) formalized
innovation as a probabilistic flow of creative destruction. In
continuous time, knowledge accumulation follows an arrival process

\[
\frac{\dot{A}}{A} = \lambda n,
\]

where (\(n\)) denotes the share of labor in R\&D and (\(\lambda\)) is
the exogenous arrival rate of successful innovation. Feasibility in this
system is \textbf{granted by assumption}: a steady balance between
creation and destruction ensures positive growth. The economy grows not
because feasibility is internally generated, but because exogenous
parameters---innovation efficiency, entry rate, and product
replacement---maintain a tractable steady state.

Formally, the model assumes that at each instant, there exists a
feasible path (\(A_t\)) satisfying the Poisson law of innovation.
Conceptually, it turns \emph{innovation} into the narrative engine of
progress while leaving the \emph{consistency of the innovation process
itself} outside the model. The system speaks of change yet presupposes
the stability that makes change possible.

\begin{quote}
The Schumpeterian system internalized the cause of growth but not the
logic of its own feasibility.
\end{quote}

\begin{center}\rule{0.5\linewidth}{0.5pt}\end{center}

A similar closure emerges in the institutional dynamics of Acemoglu and
collaborators. Whether in \emph{Why Nations Fail} or in the \emph{Race
Between Man and Machine}, feasibility of reform or adaptation appears as
an \textbf{equilibrium constraint} within a game of persistence, not as
a self-generating structure. Institutions adjust through comparative
statics of equilibrium configurations---each feasible by design, none
derived from the system's internal consistency.

Political or technological equilibria thus display what may be called
\textbf{parametric feasibility}: their stability arises from assumed
parameters---redistribution elasticity, innovation responsiveness, or
political entry thresholds---whose values secure equilibrium existence
but remain externally fixed. The dynamic system evolves within
parameters that guarantee its continuation, yet never interrogates how
these parameters could emerge endogenously.

\begin{quote}
Acemoglu's equilibrium narratives describe the persistence of order;
they do not generate the geometry of order itself.
\end{quote}

\begin{center}\rule{0.5\linewidth}{0.5pt}\end{center}

Together, these frameworks mark the formal boundary of narrative
macroeconomics: growth is fully internalized as a story of agents and
incentives, yet the system's own feasibility remains externally
parameterized. It is here that the transition must occur---from
\textbf{parametric feasibility} to \textbf{constraint geometry}, where
the conditions of motion are generated, not assumed.

\begin{center}\rule{0.5\linewidth}{0.5pt}\end{center}

\subsection{3. Toward Constraint Geometry and Dynamic
Feasibility}\label{toward-constraint-geometry-and-dynamic-feasibility}

\begin{center}\rule{0.5\linewidth}{0.5pt}\end{center}

\subsubsection{Notation and Definitions}\label{notation-and-definitions}

Let the feasible manifold \(\mathcal{F}\subseteq\mathbb{R}^n\) represent
the joint satisfaction of informational, incentive, and institutional
constraints.\\
For any \(A\in\mathcal{F}\):

\begin{itemize}
\tightlist
\item
  \(T_A(\mathcal{F})\) denotes the tangent space of feasible
  deformations.
\item
  \(N_A(\mathcal{F})\) denotes the normal space orthogonal to
  \(T_A(\mathcal{F})\).
\item
  \(\Pi_{T_A(\mathcal{F})}\) and \(\Pi_{N_A(\mathcal{F})}\) are the
  projection operators onto these subspaces.
\item
  \(\nabla\) denotes a connection compatible with the tangent bundle of
  \(\mathcal{F}\).
\end{itemize}

Unless otherwise stated, all trajectories \(A_t\) are assumed
continuously differentiable.

\begin{center}\rule{0.5\linewidth}{0.5pt}\end{center}

To surpass this boundary, one must reconstruct the dynamics under
\textbf{constraint geometry}.\\
Instead of

\[
\dot{A} = f(A, \theta)
\]

where \(\theta\) encodes exogenous feasibility,\\
the system should define a \textbf{feasibility manifold} \(\mathcal{F}\)
such that

\[
\dot{A} \in T_A(\mathcal{F})
\]

and \(\mathcal{F}\) itself evolves through the internal consistency of
the mechanism.

Feasibility thus becomes \emph{endogenous to structure}:

\begin{itemize}
\tightlist
\item
  Innovation is no longer an exogenous driver but a \emph{constraint
  deformation process}.\\
\item
  Obsolescence represents not decay but \emph{geometric
  reconfiguration}.\\
\item
  Regret minimization replaces welfare maximization as the system's
  self-stabilizing principle.
\end{itemize}

This reformulation transforms the growth narrative into a
\textbf{structurally closed generative system}---a geometry of dynamic
feasibility.

\begin{center}\rule{0.5\linewidth}{0.5pt}\end{center}

\subsubsection{\texorpdfstring{\emph{Technical Remark: Why
``Manifold''?}}{Technical Remark: Why ``Manifold''?}}\label{technical-remark-why-manifold}

The use of \emph{manifold} is not a decorative abstraction but a
structural necessity.\\
A feasible set \(\mathcal{F}\) defined as a mere subset of the state
space expresses \textbf{static admissibility}---which combinations of
variables can coexist.\\
A \emph{manifold}, by contrast, expresses \textbf{local continuity and
dynamic coherence}: the idea that feasible configurations can
\emph{evolve smoothly} without external re-specification of constraints.

Formally, if feasibility is represented by a manifold
\(\mathcal{F}\subseteq \mathbb{R}^n\), the inclusion

\[
\dot{A}\in T_A(\mathcal{F})
\]

states that the direction of change at any feasible point (A) lies
within the tangent space of the feasible structure itself.\\
This condition encodes \textbf{endogenous feasibility}: the mechanism's
evolution does not require an exogenous update of constraints; it stays
consistent within its own geometry.

In this sense, the manifold notation is \emph{syntactic}, not
computational.\\
It does not invite curvature or connection calculations, but ensures
that the formalism respects \textbf{self-sustaining continuity}---the
hallmark of generative systems.

\begin{quote}
A set constrains choice;\\
a manifold sustains motion.
\end{quote}

\begin{center}\rule{0.5\linewidth}{0.5pt}\end{center}

\subsubsection{\texorpdfstring{3.1 \textbf{Reinterpreting Dynamics: From
Drivers to Constraint
Deformations}}{3.1 Reinterpreting Dynamics: From Drivers to Constraint Deformations}}\label{reinterpreting-dynamics-from-drivers-to-constraint-deformations}

Having recognized that in parameterized narratives feasibility is
guaranteed by external parameters, we now reconstruct dynamics where
feasibility becomes \emph{endogenous to structure}. Dynamics cease to be
the effect of exogenous impulses; they are the infinitesimal
reconfigurations of the constraint surface itself.

\[
\dot{A} \in T_A(\mathcal{F})
\]

denotes not the movement of an external variable but the system's local
adherence to its feasible manifold. The structure's curvature evolves as
agents adjust within their internal limitations, yielding three
interpretive shifts:

\begin{center}\rule{0.5\linewidth}{0.5pt}\end{center}

\textbf{(1) Innovation as constraint deformation.}\\
In the Schumpeterian narrative, innovation enters as an external
driver---through an exogenous rate \(\lambda\) and an allocative share
\(n\)---that ensures steady accumulation:

\[
\frac{\dot{A}}{A} = \lambda n.
\]

In the geometric form, by contrast, innovation appears as a local
deformation of the feasibility manifold \(\mathcal{F}\). Each
reconfiguration of incentive, knowledge, or institutional consistency
redefines what counts as a feasible direction of motion.\\
Growth is not \emph{caused} by innovation; it \emph{is} the geometry of
constraint adjustment.

\begin{center}\rule{0.5\linewidth}{0.5pt}\end{center}

\textbf{(2) Obsolescence as reconfiguration.}\\
What conventional narratives call technological decay or creative
destruction is, under the geometric lens, a \emph{topological
transition} of (\(\mathcal{F}\)).\\
Obsolescence is not a loss of knowledge but the folding or unfolding of
feasible domains: old regions collapse, new ones emerge. The system does
not deteriorate; it rewrites its own coordinate chart.

\begin{center}\rule{0.5\linewidth}{0.5pt}\end{center}

\textbf{(3) Regret minimization as self-stabilization.}\\
Absent an external welfare function, systemic stability arises from
internal coherence.\\
Each agent minimizes local \emph{regret}---the infinitesimal
inconsistency between its actual motion and the feasible tangent
direction prescribed by the structure.

\begin{Shaded}
\begin{Highlighting}[]
\NormalTok{\textbackslash{}min\_t \textbackslash{}| \textbackslash{}dot\{A\}\_t {-} \textbackslash{}Pi\_\{T\_\{A\_t\}(\textbackslash{}mathcal\{F\})\}(\textbackslash{}dot\{A\}\_t) \textbackslash{}|,}
\end{Highlighting}
\end{Shaded}

where \(|\cdot|\) denotes any norm consistent with the inner product on
the ambient space.

This local alignment replaces welfare maximization: the system persists
not by achieving global optima but by maintaining tangent
adherence---remaining self-consistent under deformation.

\begin{center}\rule{0.5\linewidth}{0.5pt}\end{center}

\textbf{Hence:}\\
Innovation, obsolescence, and stabilization are no longer external
episodes within a narrative of progress.\\
They are geometric operations within a single constraint
surface---deformation, reconfiguration, and projection.\\
The system evolves by staying feasible.

\begin{center}\rule{0.5\linewidth}{0.5pt}\end{center}

\subsubsection{\texorpdfstring{3.2 \textbf{Formal Extension: Tangent
Projection and Constraint
Preservation}}{3.2 Formal Extension: Tangent Projection and Constraint Preservation}}\label{formal-extension-tangent-projection-and-constraint-preservation}

Let the feasible manifold \(\mathcal{F} \subset \mathbb{R}^n\) represent
the endogenous consistency of a mechanism---its joint satisfaction of
informational, incentive, and institutional constraints.\\
Each state \(A_t \in \mathcal{F}\) evolves through time according to an
internal dynamic \(\dot{A}_t\), but this dynamic must remain
\emph{tangent-consistent}: the motion should not depart from the
feasible surface.

Define the orthogonal decomposition:

\[
\dot{A}*t = \Pi*{T_{A_t}(\mathcal{F})}(\dot{A}*t) + \Pi*{N_{A_t}(\mathcal{F})}(\dot{A}_t),
\]

where \(T_{A_t}(\mathcal{F})\) is the tangent space of feasible
deformations, and \(N_{A_t}(\mathcal{F})\) the normal space representing
infeasible impulses.\\
A dynamically feasible path satisfies:

\[
\Pi_{N_{A_t}(\mathcal{F})}(\dot{A}_t) = 0.
\]

When external perturbations or strategic deviations push the system away
from feasibility, a \emph{restorative projection} occurs:

\[
\dot{A}*t^{\ast} = \Pi*{T_{A_t}(\mathcal{F})}(\dot{A}_t),
\]

interpretable as the system's endogenous correction mechanism---its
\emph{regret response}.

\begin{center}\rule{0.5\linewidth}{0.5pt}\end{center}

\paragraph{\texorpdfstring{\textbf{Connection Form and
Curvature}}{Connection Form and Curvature}}\label{connection-form-and-curvature}

The evolution of \(T_{A_t}(\mathcal{F})\) itself is governed by a
connection form \(\nabla\), encoding how local feasibility directions
twist as the constraint structure deforms.\\
For any tangent vector fields \(u,v \in T(\mathcal{F})\) along a smooth
trajectory \(A_t\):

\[
\nabla_{\dot{A}*t} v = \Pi*{T_{A_t}(\mathcal{F})} ( Dv[\dot{A}_t] ),
\]

ensuring that differentiation respects the constraint geometry.

The curvature

\[
R(u,v) = \nabla_u\nabla_v - \nabla_v\nabla_u - \nabla_{[u,v]},\quad u,v \in T_A(\mathcal{F}).
\]

then measures how local incentive constraints fail to commute---how the
order of adaptation matters.\\
Regions of high curvature correspond to \emph{structural tension}:
domains where incentive or belief adjustments cannot be globally
reconciled.\\
In equilibrium theory, such curvature often manifests as cyclical
instability; here it encodes the \emph{irreducible tension} that defines
the structure's internal generativity.

\begin{center}\rule{0.5\linewidth}{0.5pt}\end{center}

\paragraph{\texorpdfstring{\textbf{Geometric Restatement of
Stability}}{Geometric Restatement of Stability}}\label{geometric-restatement-of-stability}

The system's stability condition may be expressed as a bounded deviation
from tangent feasibility:

\[
|\Pi_{N_{A_t}(\mathcal{F})}(\dot{A}_t)| < \varepsilon,
\]

where \(\varepsilon\) quantifies the tolerance of the mechanism to
inconsistency.\\
The limit \(\varepsilon \to 0\) defines \textbf{structural equilibrium}:
every motion is feasible, every deformation is self-consistent.

Under this interpretation, stability is not a fixed point but a
\emph{persistent alignment}:

\begin{quote}
A mechanism is stable not because it rests, but because it continues to
move along its manifold.
\end{quote}

\begin{center}\rule{0.5\linewidth}{0.5pt}\end{center}

\paragraph{\texorpdfstring{\textbf{Interpretive
Remark}}{Interpretive Remark}}\label{interpretive-remark}

This differential formalism turns ``growth'' into the infinitesimal
evolution of a constraint surface.\\
Innovation corresponds to curvature shifts, obsolescence to manifold
reparameterization, and regret minimization to tangent projection.\\
Each mechanism is thus a self-sustaining geometric process:\\
it evolves \emph{by maintaining its own feasibility}.

\begin{center}\rule{0.5\linewidth}{0.5pt}\end{center}

\subsection{4. From Endogenous Growth to Generative
Stability}\label{from-endogenous-growth-to-generative-stability}

Aghion and Acemoglu completed the second rupture: the internalization of
innovation and institutions.\\
The next rupture must internalize \textbf{feasibility itself}.

Where their models end in narrative equilibrium, the
constraint-geometric framework begins as a \textbf{structural
equilibrium}:\\
a configuration where dynamics sustain their own conditions of
continuation.

\begin{quote}
The question is no longer ``how innovation drives growth,''\\
but ``under what constraint geometry innovation remains feasible.''
\end{quote}

This shift---from parametric equilibrium to generative stability---marks
the transition from narrative macroeconomics to structural theory.

\begin{center}\rule{0.5\linewidth}{0.5pt}\end{center}

\subsection{References}\label{references}

\begin{itemize}
\tightlist
\item
  Aghion, Philippe \& Howitt, Peter (1992). \emph{A Model of Growth
  through Creative Destruction}.\\
\item
  Acemoglu, Daron \& Robinson, James (2012). \emph{Why Nations Fail}.\\
\item
  Acemoglu, Daron \& Restrepo, Pascual (2018). \emph{The Race Between
  Man and Machine}.\\
\item
  Smale (1976) \emph{Mathematical Economics and General Equilibrium
  Theory.}
\item
  Marsden \& Ratiu (1999) \emph{Introduction to Mechanics and Symmetry}
\end{itemize}

\begin{center}\rule{0.5\linewidth}{0.5pt}\end{center}

\subsubsection{\texorpdfstring{\textbf{Appendix: Relation to Hamiltonian
Systems}}{Appendix: Relation to Hamiltonian Systems}}\label{appendix-relation-to-hamiltonian-systems}

The constraint--geometric formulation may be viewed as a degenerate
limit of Hamiltonian dynamics under total dissipation of symplectic
structure.

In a Hamiltonian system, evolution follows a \emph{symplectic flow}:

\begin{Shaded}
\begin{Highlighting}[]
\NormalTok{\textbackslash{}dot\{x\} = J \textbackslash{}nabla H(x),}
\end{Highlighting}
\end{Shaded}

where the antisymmetric operator \(J\) preserves a two-form
\(\omega = dx \wedge J dx\); energy is conserved, and stability is
defined by invariance of \(H\).

By contrast, the constraint--geometric system replaces symplectic
invariance with \emph{feasibility invariance}:

\begin{Shaded}
\begin{Highlighting}[]
\NormalTok{\textbackslash{}dot\{A\} \textbackslash{}in T\_A(\textbackslash{}mathcal\{F\}).}
\end{Highlighting}
\end{Shaded}

No energy function \(H\) governs motion; instead, the system preserves
the local geometry of constraints. Where Hamiltonian flows are
\emph{conservative}, constraint flows are \emph{projective}: they
restore consistency by eliminating the normal component rather than
balancing momenta.

This exchange of invariants---energy for feasibility---marks the
conceptual boundary between physics and mechanism. A Hamiltonian system
sustains motion by conserving form across time; a constraint--geometric
system sustains form by continuously redrawing what counts as feasible
time.

\begin{quote}
\emph{The former sustains motion by coherence in space; the latter, by
coherence in structure.}
\end{quote}

\begin{center}\rule{0.5\linewidth}{0.5pt}\end{center}

\emph{Residue: Curvature as Structural Tension in Dynamic Mechanism}

\emph{Footnote}: This note deliberately avoids introducing a welfare
functional. The focus is on the feasibility geometry of motion rather
than optimization with respect to preferences.

\bibliographystyle{apalike}
\bibliography{refs}

\end{document}
